\chapter{Introduction}
\label{chapter:intro}

Introduction of the thesis. Sample citation \cite{comparing}. Sample acronym: \ac{UPM} 



The present thesis aims to contribute to the growing field of *** by answering the following \acp{RQ}:
 
\begin{enumerate}
    \item \label{rq:one} \textit{\Copy{rqone}{What is my first research question?}}
    \item \label{rq:two} \textit{\Copy{rqtwo}{What is my second research question?}}
\end{enumerate}



\section{Objectives}

The main aim of this thesis is to contribute to the understanding of ****. In order to make the aforementioned contributions, the following objectives have been formulated:

\begin{enumerate}
    \item \label{obj:one} \textit{\Copy{objone}{Objective 1 of the thesis.}}
    
    \item \label{obj:two} \textit{\Copy{objtwo}{Objective 2 of the thesis}}
    
    
\end{enumerate}

Lastly, based on the achievement of these objectives, this thesis aims to answer the research questions previously stated.

\section{Research methodology}

This section describes the methodology followed in this thesis in order to achieve the specific goals previously stated and answer the research questions.

\section{Structure of this document}

\noindent \textbf{Chapter \ref{chapter:rw}: \nameref{chapter:rw}}

\vspace{-3mm}
\noindent This chapter provides an introduction to the field of ***. Objective \ref{obj:one} of this thesis is addressed in this chapter. In addition, \acp{RQ} \ref{rq:one} is answered in this chapter.
 
\noindent \textbf{Chapter \ref{chapter:something}: \nameref{chapter:something}}

\vspace{-3mm}
\noindent  This chapter reports on  Objective \ref{obj:two} and  \ac{RQ} \ref{rq:two} are addressed in this chapter. 

...

\noindent \textbf{Chapter \ref{chapter:validation}: \nameref{chapter:validation}}

\vspace{-3.5mm}
\noindent This chapter briefly describes the projects and learning experiences through which the contributions of this thesis have been validated, and lists the different publications that have been produced as a result of it. Furthermore, this chapter exhibits the contribution of this thesis to the research and educational communities through open-source software.


\noindent \textbf{Chapter \ref{chapter:conclusions}: \nameref{chapter:conclusions}}

\vspace{-3.5mm}
\noindent This chapter concludes the thesis with a summary of the answers to the research questions, a synthesis of the main contributions and some suggestions for further research. \noclub